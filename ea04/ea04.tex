\documentclass[usenames,dvipsnames,aspectratio=169]{beamer}
\usepackage{../common/cpp}

\title[OO Programozás - C++]{OO Programozás}
\subtitle{Operátor felültöltés, másolás és átalakítás}

\begin{document}

%1
\begin{frame}[plain]
  \titlepage
  \logoalul
\end{frame}

\section{Függvénysablonok}

\begin{frame}
    Sablonok (template)
    \begin{itemize}
        \item Különböző típusokat kezelő függvények / osztályok csoportjának egyszerű létrehozása
        \item \emph{Típusparaméter} (fordítási időben eldől) vs. függvény paraméter (futási időben adják át)
        \item Kulcsszó: \texttt{class} vagy \texttt{typename} (\emph{majdnem} egyenértékűek)
        \item \emph{Sablon példányosítás:} adott típusparaméterrel új változat előállítása $\to$ egyszerűbb, mint sok felültöltött függvényt kézzel létrehozni
    \end{itemize}
    \vfill
    \begin{description}[m]
        \item[Feladat:] \hfill \\ Készítsünk \texttt{max} függvényt, ami két paramétere közül visszaadja a nagyobbat!
    \end{description}
\end{frame}

\begin{frame}
    \begin{exampleblock}{\textattachfile{max1.cpp}{max1.cpp}}
        \lstinputlisting[language=C++,linerange={3-10},numbers=left,firstnumber=3]{max1.cpp}
    \end{exampleblock}
\end{frame}

\begin{frame}[fragile]
    \begin{exampleblock}{\textattachfile{max1.cpp}{max1.cpp}}
        \vspace{-.2cm}
        \scriptsize
        \lstinputlisting[language=C++,linerange={12-22},numbers=left,firstnumber=12]{max1.cpp}
        \vspace{-.2cm}
    \end{exampleblock}
    \begin{block}{Kimenet}
        \vspace{-.4cm}
        \scriptsize
        \begin{verbatim}
1 es 2 kozul 2 a nagyobb.
1.5 es 2.5 kozul 2.5 a nagyobb.
\end{verbatim}             
        \vspace{-.4cm}
    \end{block}
\end{frame}

\begin{frame}
    \begin{exampleblock}{\textattachfile{max2.cpp}{max2.cpp}}
        \scriptsize
        \lstinputlisting[language=C++,linerange={3-16},numbers=left,firstnumber=3]{max2.cpp}
    \end{exampleblock}
\end{frame}

\begin{frame}
    \begin{itemize}
        \item Egy sablon több típusparamétert is fogadhat, de akkor ezeket mind használni is kell!
        \item Csak akkor állít elő kódot a fordító, amikor a függvény konkrét hívásával találkozik $\to$ a példányosításig a hibás sablon nem vált ki hibaüzenetet
        \item Pontosan olyan változatok jönnek létre, amikre szükség is van
        \item Típusonként pontosan egy változatot hoz létre, akkor is, ha létezik a típusok között implicit konverzió (pl. \texttt{int} $\to$ \texttt{long})
        \item Mi történik, ha két \emph{eltérő} típusú paraméterrel hívják a függvényünket?
    \end{itemize}
\end{frame}

\begin{frame}
    \begin{exampleblock}{\textattachfile{max2.cpp}{max2.cpp}}
        \scriptsize
        \lstinputlisting[language=C++,linerange={18-27},numbers=left,firstnumber=18]{max2.cpp}
    \end{exampleblock}
\end{frame}

\begin{frame}
    \begin{description}[m]
        \item[Probléma:] \hfill \\ A két paraméter típusának mindenképpen egyeznie kell!
        \item[Félmegoldás:] \hfill \\ Két típusparaméter használata (nem befolyásolható futásidőben, melyikben keletkezzen az eredmény)  
    \end{description}
\end{frame}

\begin{frame}[fragile]
    \begin{columns}[T]
        \column{0.65\textwidth}
            \begin{exampleblock}{\textattachfile{max3.cpp}{max3.cpp}}
                \scriptsize
                \lstinputlisting[language=C++,linerange={3-14},numbers=left,firstnumber=3]{max3.cpp}
            \end{exampleblock}
        \column{0.35\textwidth}
            \begin{block}{Kimenet}
                \footnotesize
                \vspace{-.4cm}
                \begin{verbatim}
3 es 4.5 kozul 4 a nagyobb.
4.5 es 3 kozul 4.5 a nagyobb.
                \end{verbatim}
                \vspace{-.4cm}
            \end{block}
    \end{columns}
\end{frame}

\begin{frame}[fragile]
    Ha létezik függvény a szükséges paraméterekkel, nem jön létre példány a sablonból, és a korábbi hiba orvosolható.
    \begin{exampleblock}{\textattachfile{max4.cpp}{max4.cpp}}
        \scriptsize
        \lstinputlisting[language=C++,linerange={3-9},numbers=left,firstnumber=3]{max4.cpp}
    \end{exampleblock}
    \begin{block}{Kimenet}
        \footnotesize
        \vspace{-.4cm}
        \begin{verbatim}
3 es 4.5 kozul 4.5 a nagyobb.
4.5 es 3 kozul 4.5 a nagyobb.
        \end{verbatim}
        \vspace{-.4cm}
    \end{block}
\end{frame}

\section{Osztálysablonok}

\end{document}