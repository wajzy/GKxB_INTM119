\documentclass[usenames,dvipsnames,aspectratio=169]{beamer}
\usepackage{../common/cpp}

\title[OO Programozás - C++]{OO Programozás}
\subtitle{Sablonok, kivételkezelés}

\begin{document}

%1
\begin{frame}[plain]
  \titlepage
  \logoalul
\end{frame}

\section{Sablonok}

\subsection{Függvénysablonok}

\begin{frame}
    Sablonok (template)
    \begin{itemize}
        \item Különböző típusokat kezelő függvények / osztályok csoportjának egyszerű létrehozása
        \item \emph{Típusparaméter} (fordítási időben eldől) vs. függvény paraméter (futási időben adják át)
        \item Kulcsszavak:
        \begin{itemize}
            \item Sablon $\to$ \texttt{template}
            \item Típusparaméterek $\to$ \texttt{class} vagy \texttt{typename} (\emph{majdnem} egyenértékűek)
        \end{itemize}
        \item \emph{Sablon példányosítás:} adott típusparaméterrel új változat előállítása $\to$ egyszerűbb, mint sok felültöltött függvényt kézzel létrehozni
    \end{itemize}
    \vfill
    \begin{description}[m]
        \item[Feladat:] \hfill \\ Készítsünk \texttt{max} függvényt, ami két paramétere közül visszaadja a nagyobbat!
    \end{description}
\end{frame}

\begin{frame}
    \begin{exampleblock}{\textattachfile{max1.cpp}{max1.cpp}}
        \lstinputlisting[language=C++,linerange={3-10},numbers=left,firstnumber=3]{max1.cpp}
    \end{exampleblock}
\end{frame}

\begin{frame}[fragile]
    \begin{exampleblock}{\textattachfile{max1.cpp}{max1.cpp}}
        \vspace{-.2cm}
        \scriptsize
        \lstinputlisting[language=C++,linerange={12-22},numbers=left,firstnumber=12]{max1.cpp}
        \vspace{-.2cm}
    \end{exampleblock}
    \begin{block}{Kimenet}
        \vspace{-.4cm}
        \scriptsize
        \begin{verbatim}
1 es 2 kozul 2 a nagyobb.
1.5 es 2.5 kozul 2.5 a nagyobb.
\end{verbatim}             
        \vspace{-.4cm}
    \end{block}
\end{frame}

\begin{frame}
    \begin{exampleblock}{\textattachfile{max2.cpp}{max2.cpp}}
        \scriptsize
        \lstinputlisting[language=C++,linerange={3-16},numbers=left,firstnumber=3]{max2.cpp}
    \end{exampleblock}
\end{frame}

\begin{frame}
    \begin{itemize}
        \item Egy sablon több típusparamétert is fogadhat, de akkor ezeket mind használni is kell!
        \item Csak akkor állít elő kódot a fordító, amikor a függvény konkrét hívásával találkozik $\to$ a példányosításig a hibás sablon nem vált ki hibaüzenetet
        \item Pontosan olyan változatok jönnek létre, amikre szükség is van
        \item Típusonként pontosan egy változatot hoz létre, akkor is, ha létezik a típusok között implicit konverzió (pl. \texttt{int} $\to$ \texttt{long})
        \item Mi történik, ha két \emph{eltérő} típusú paraméterrel hívják a függvényünket?
    \end{itemize}
\end{frame}

\begin{frame}
    \begin{exampleblock}{\textattachfile{max2.cpp}{max2.cpp}}
        \scriptsize
        \lstinputlisting[language=C++,linerange={18-27},numbers=left,firstnumber=18]{max2.cpp}
    \end{exampleblock}
\end{frame}

\begin{frame}
    \begin{description}[m]
        \item[Probléma:] \hfill \\ A két paraméter típusának mindenképpen egyeznie kell!
        \item[Félmegoldás:] \hfill \\ Két típusparaméter használata (nem befolyásolható futásidőben, melyikben keletkezzen az eredmény)  
    \end{description}
\end{frame}

\begin{frame}[fragile]
    \begin{columns}[T]
        \column{0.65\textwidth}
            \begin{exampleblock}{\textattachfile{max3.cpp}{max3.cpp}}
                \scriptsize
                \lstinputlisting[language=C++,linerange={3-14},numbers=left,firstnumber=3]{max3.cpp}
            \end{exampleblock}
        \column{0.35\textwidth}
            \begin{block}{Kimenet}
                \footnotesize
                \vspace{-.4cm}
                \begin{verbatim}
3 es 4.5 kozul 4 a nagyobb.
4.5 es 3 kozul 4.5 a nagyobb.
                \end{verbatim}
                \vspace{-.4cm}
            \end{block}
    \end{columns}
\end{frame}

\begin{frame}[fragile]
    Ha létezik függvény a szükséges paraméterekkel, nem jön létre példány a sablonból, és a korábbi hiba orvosolható.
    \begin{exampleblock}{\textattachfile{max4.cpp}{max4.cpp}}
        \scriptsize
        \lstinputlisting[language=C++,linerange={3-9},numbers=left,firstnumber=3]{max4.cpp}
    \end{exampleblock}
    \begin{block}{Kimenet}
        \footnotesize
        \vspace{-.4cm}
        \begin{verbatim}
3 es 4.5 kozul 4.5 a nagyobb.
4.5 es 3 kozul 4.5 a nagyobb.
        \end{verbatim}
        \vspace{-.4cm}
    \end{block}
\end{frame}

\subsection{Osztálysablonok}

\begin{frame}
    Fontosabb tudnivalók:
    \begin{itemize}
        \item A függvénysablonokhoz hasonlóan \emph{osztálysablonokat} is létrehozhatunk.
        \item A típusparaméterek mellett \emph{konstansparamétereket} is megadhatunk $\to$ bárhova helyettesíthetők, ahol konstans kifejezés állhat.
        \item Ha egy tagfüggvényt az osztályon kívül definiálunk, meg kell ismételni a \texttt{template} kulcsszót a paraméterekkel együtt. A definíciónak kivételesen a fejfájlban a helye, hogy a fordító mindig elérje.
        \item Osztálysablon példányosítása: a típus után \texttt{<} és \texttt{>} között meg kell adni a konkrét típusokat, konstansokat. Csak ezek együtt azonosítják egyértelműen a sablon alapján előállított osztályt!
    \end{itemize}
\end{frame}

\begin{frame}
    \begin{exampleblock}{\textattachfile{Stack1.h}{Stack1.h}}
        \lstinputlisting[language=C++,linerange={6-15},numbers=left,firstnumber=6]{Stack1.h}
    \end{exampleblock}
\end{frame}

\begin{frame}
    \begin{exampleblock}{\textattachfile{Stack1.h}{Stack1.h}}
        \small
        \lstinputlisting[language=C++,linerange={16-27},numbers=left,firstnumber=16]{Stack1.h}
    \end{exampleblock}
\end{frame}

\begin{frame}
    \begin{exampleblock}{\textattachfile{Stack1.h}{Stack1.h}}
        \lstinputlisting[language=C++,linerange={29-36},numbers=left,firstnumber=29]{Stack1.h}
    \end{exampleblock}
\end{frame}

\begin{frame}
    \begin{exampleblock}{\textattachfile{stackMain1.cpp}{stackMain1.cpp}}
        \lstinputlisting[language=C++,linerange={1-10},numbers=left,firstnumber=1]{stackMain1.cpp}
    \end{exampleblock}
\end{frame}

\begin{frame}
    \begin{exampleblock}{\textattachfile{stackMain1.cpp}{stackMain1.cpp}}
        \lstinputlisting[language=C++,linerange={12-17},numbers=left,firstnumber=12]{stackMain1.cpp}
    \end{exampleblock}
\end{frame}

\section{Kivételkezelés}

\subsection{Általános szabályok}

\begin{frame}
    Kivételkezelés (\emph{exception handling})
    \begin{itemize}
        \item \emph{Kivételes}, azaz ritka események (hibák) kezelésére
        \item Előidézheti hardveres (pl. elfogy a memória) vagy szoftveres (pl. dinamikus tömb túlindexelése) hiba
        \item Védett blokk: felkészülés kivételes helyzet bekövetkezésére $\to$ \texttt{try}
        \item Kivétel kiváltása $\to$ \texttt{throw}
        \item Kivétel elfogása $\to$ \texttt{catch}
    \end{itemize}
\end{frame}

\begin{frame}
    \texttt{try} blokkok jellemzői:
    \begin{itemize}
        \item Egymásba ágyazhatók.
        \item Benne közvetlenül vagy közvetetten (függvényen belül) is használható \texttt{throw} utasítás.
        \item Több \texttt{catch} ág is tartozhat egy \texttt{try}-hoz, vagy akár egy sem.
    \end{itemize}
    \vfill
    \texttt{throw} jellemzői:
    \begin{itemize}
        \item Tetszőleges típusú kivétel kiváltható.
        \item A \texttt{try} blokk végrehajtása azonnal félbeszakad a hatására.
        \item Használható akár konstruktorban, vagy \texttt{catch} blokkban is.
        \item A \texttt{throw;} utasítás a korábban kiváltott és már elfogott típusú és tartalmú kivétellel azonosat vált ki újra.
    \end{itemize}
\end{frame}

\begin{frame}
    \texttt{catch} blokkok jellemzői:
    \begin{itemize}
        \item Az első illeszkedő blokk kezeli a kivételt.
        \item Nyilvánosan származtatott kivétel osztályok esetén a szülő típusára illeszthető a leszármazott típusú kivétel $\to$ speciálisak előre, általánosabbak hátra!
        \item Nem illeszkednek viszont az implicit típuskonverzióval egyébként egymásba alakítható típusok \texttt{catch} blokkjai (pl. \texttt{int} típusú kivételt nem fogja el a \texttt{float} kezelője).
        \item A speciális, \texttt{\dots} típusú \texttt{catch} minden, korábban nem kezelt kivételt elfog $\to$ ez legyen az utolsó!
        \item A kivétel kezelését követően a program fut tovább az utolsó \texttt{catch} blokk után.
        \item Ha nincs illeszthető \texttt{catch} ág, a vezérlés a befoglaló \texttt{try-catch} szerkezetre kerül, és ott keres illeszkedő kivételkezelőt. Ha ilyen nincs, a vezérlés a \texttt{terminate()} függvényre kerül, és a program leáll.
    \end{itemize}
\end{frame}

\begin{frame}
    Kivételkezelési stratégiák:
    \begin{enumerate}
        \item Figyelmeztetés, felhasználói beavatkozás kérése, megpróbálkozás a hiba javításával.
        \item Program leállítása (pl. \texttt{exit()}, \texttt{abort()}) súlyos hibák esetén.
        \item Kivétel kezeletlenül hagyása.
    \end{enumerate}
    \vfill
    A dinamikusan foglalt erőforrásokat szükség esetén fel kell szabadítani!
\end{frame}

\begin{frame}
    \begin{exampleblock}{\textattachfile{trycatch1.cpp}{trycatch1.cpp}}
        \lstinputlisting[language=C++,linerange={1-11},numbers=left,firstnumber=1]{trycatch1.cpp}
    \end{exampleblock}
\end{frame}

\begin{frame}
    \begin{exampleblock}{\textattachfile{trycatch1.cpp}{trycatch1.cpp}}
        \small
        \lstinputlisting[language=C++,linerange={12-23},numbers=left,firstnumber=12]{trycatch1.cpp}
    \end{exampleblock}
\end{frame}

\begin{frame}
    \begin{exampleblock}{\textattachfile{trycatch1.cpp}{trycatch1.cpp}}
        \small
        \lstinputlisting[language=C++,linerange={24-36},numbers=left,firstnumber=24]{trycatch1.cpp}
    \end{exampleblock}
\end{frame}

\begin{frame}
    \begin{exampleblock}{\textattachfile{trycatch1.cpp}{trycatch1.cpp}}
        \small
        \lstinputlisting[language=C++,linerange={37-45},numbers=left,firstnumber=37]{trycatch1.cpp}
    \end{exampleblock}
\end{frame}

\begin{frame}
    \begin{exampleblock}{\textattachfile{trycatch1.cpp}{trycatch1.cpp}}
        \small
        \lstinputlisting[language=C++,linerange={46-56},numbers=left,firstnumber=46]{trycatch1.cpp}
    \end{exampleblock}
\end{frame}

\begin{frame}[fragile]
    \begin{block}{Kimenet}
        \vspace{-.3cm}
        \begin{verbatim}
Throwing int... caught an UNKNOWN exception
Throwing double... caught 3.14
Throwing const char*... caught an unknown message
Throwing BaseException... caught BaseException object
Throwing DerivedException... caught BaseException object
Throwing char... caught x, re-throw exception
terminate called after throwing an instance of 'char'
Aborted (core dumped)            
\end{verbatim}
    \vspace{-.3cm}
    \end{block}
\end{frame}

\begin{frame}
    \begin{exampleblock}{\textattachfile{trycatch2.cpp}{trycatch2.cpp}}
        \footnotesize
        \lstinputlisting[language=C++,linerange={3-13},numbers=left,firstnumber=3]{trycatch2.cpp}
    \end{exampleblock}
\end{frame}

\begin{frame}
    \begin{exampleblock}{\textattachfile{trycatch2.cpp}{trycatch2.cpp}}
        \vspace{-.2cm}
        \fontsize{8}{9} \selectfont
        \lstinputlisting[language=C++,linerange={15-32},numbers=left,firstnumber=15]{trycatch2.cpp}
        \vspace{-.2cm}
    \end{exampleblock}
\end{frame}

\begin{frame}[fragile]
    \begin{block}{Kimenet1}
        \vspace{-.45cm}
        \small
        \begin{verbatim}
Do you want to throw an exception of type int? y
Caught an int.
\end{verbatim}
    \vspace{-.45cm}
    \end{block}
    \begin{block}{Kimenet2}
        \vspace{-.45cm}
        \small
        \begin{verbatim}
Do you want to throw an exception of type int? n
Caught a double.
Do you want to throw an exception of type char? y
Caught a char.
\end{verbatim}
    \vspace{-.45cm}
    \end{block}
    \begin{block}{Kimenet3}
        \vspace{-.45cm}
        \small
        \begin{verbatim}
Do you want to throw an exception of type int? n
Caught a double.
Do you want to throw an exception of type char? n
Caught a pointer to char.
\end{verbatim}
    \vspace{-.45cm}
    \end{block}
\end{frame}

\subsection{A verem mintafeladat átalakítása}

\begin{frame}
    Alakítsuk át a verem megvalósítását úgy, hogy kivételkezelést használjon! Használjuk fel a függvénykönyvtár \hiv{\href{https://en.cppreference.com/w/cpp/error/exception}{\texttt{exception}}} osztályát!
    \begin{exampleblock}{\textattachfile{Stack2.h}{Stack2.h}}
        \small
        \lstinputlisting[language=C++,linerange={4-12},numbers=left,firstnumber=4]{Stack2.h}
    \end{exampleblock}
\end{frame}

\begin{frame}
    \begin{exampleblock}{\textattachfile{Stack2.h}{Stack2.h}}
        \small
        \lstinputlisting[language=C++,style=C++,linerange={13-24},numbers=left,firstnumber=17]{Stack2.h}
    \end{exampleblock}
\end{frame}

\begin{frame}
    \begin{exampleblock}{\textattachfile{Stack2.h}{Stack2.h}}
        \scriptsize
        \lstinputlisting[language=C++,style=C++,linerange={26-26},numbers=left,firstnumber=26]{Stack2.h}
        \lstinputlisting[language=C++,style=C++,linerange={30-30},numbers=left,firstnumber=30]{Stack2.h}
        \lstinputlisting[language=C++,style=C++,linerange={35-45},numbers=left,firstnumber=35]{Stack2.h}
    \end{exampleblock}
\end{frame}

\begin{frame}
    \begin{exampleblock}{\textattachfile{Stack2.h}{Stack2.h}}
        \lstinputlisting[language=C++,style=C++,linerange={51-60},numbers=left,firstnumber=51]{Stack2.h}
    \end{exampleblock}
\end{frame}

\begin{frame}
    \begin{exampleblock}{\textattachfile{stackMain2.cpp}{stackMain2.cpp}}
        \footnotesize
        \lstinputlisting[language=C++,style=C++,linerange={4-16},numbers=left,firstnumber=4]{stackMain2.cpp}
    \end{exampleblock}
\end{frame}

\begin{frame}
    \begin{exampleblock}{\textattachfile{stackMain2.cpp}{stackMain2.cpp}}
        \small
        \lstinputlisting[language=C++,style=C++,linerange={18-29},numbers=left,firstnumber=18]{stackMain2.cpp}
    \end{exampleblock}
\end{frame}

\begin{frame}[fragile]
    \begin{block}{Kimenet}
        \vspace{-.45cm}
        \small
        \begin{verbatim}
Error code: 2, Sorry, stack full.
3
2
1
Hello World!
Error code: 1, Oops, the stack is empty.
\end{verbatim}
    \vspace{-.45cm}
    \end{block}
\end{frame}

\end{document}