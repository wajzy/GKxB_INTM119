\section{Származtatás, absztrakt osztályok, interfészek}

\subsection{Származtatás, védett tagok, virtuális függvények}

\begin{frame}
    \small
    Feladat: készítsünk osztályokat egy átlagos alkalmazott, és egy programozó bérének kiszámítására!
    \begin{center}
        \includegraphics[scale=0.75]{inheritance01.eps} \\
        \tiny A program \hiv{\href{https://www.visual-paradigm.com/guide/uml-unified-modeling-language/uml-class-diagram-tutorial/}{UML osztálydiagram}}ja.
    \end{center}
\end{frame}

\begin{frame}
    \begin{columns}[T]
        \column{.4\textwidth}
            \begin{exampleblock}{\textattachfile{inheritance01.cpp}{Employee}}
                \vspace{-.2cm}
                \fontsize{7}{8} \selectfont
                \lstinputlisting[language=C++,linerange={3-15},numbers=left,firstnumber=3]{inheritance01.cpp}
                \vspace{-.2cm}
            \end{exampleblock}
        \column{.6\textwidth}
            \begin{exampleblock}{\textattachfile{inheritance01.cpp}{Programmer}}
                \vspace{-.2cm}
                \fontsize{7}{8} \selectfont
                \lstinputlisting[language=C++,linerange={17-34},numbers=right,firstnumber=17]{inheritance01.cpp}
                \vspace{-.2cm}
            \end{exampleblock}
    \end{columns}
\end{frame}

\begin{frame}[fragile]
    \begin{exampleblock}{\textattachfile{inheritance01.cpp}{main()}}
        \footnotesize
        \lstinputlisting[language=C++,linerange={36-41},numbers=left,firstnumber=36]{inheritance01.cpp}
    \end{exampleblock}
    \begin{block}{Kimenet}
        \footnotesize
        \vspace{-.3cm}
        \begin{verbatim}
300000
750000
        \end{verbatim}
        \vspace{-.6cm}
    \end{block}
\end{frame}

\begin{frame}
    Probléma:
    \begin{itemize}
        \item Hasonló feladatok következménye: kódismétlés (a programozó az alkalmazott egy speciális esete).
    \end{itemize}
    Megoldás:
    \begin{itemize}
        \item Származtatás/öröklés (inheritance)
        \item Ősosztály/szülő osztály (superclass, base class) $\to$ \texttt{Employee}
        \item Leszármazott/származtatott/gyerek osztály (subclass, derived class) $\to$ \texttt{Programmer}
    \end{itemize}
\end{frame}

\begin{frame}
    \begin{center}
        \includegraphics[scale=0.75]{inheritance02.eps}
      \end{center}
\end{frame}

\begin{frame}
    \begin{itemize}
        \item A leszármazottakban (felül)definiáljuk azokat a függvényeket, amik másképpen fognak viselkedni (\texttt{getSalary()}), mint az ősben vagy teljesen hiányoztak (\texttt{setBonus()}).
        \item Minden más átöröklődik, kivéve a konstruktort. (Ettől még nem feltétlenül elérhetők ezek a leszármazottban! $\to$ \texttt{private})
        \item Ősosztálytól örökölt tagok inicializálása a leszármazott dolga, pl. az ős konstruktorának felhasználásával a taginicializáló listán. Ennek hiányában a fordító az ős alapértelmezett konstruktorát hívja.
    \end{itemize}
\end{frame}

\begin{frame}
    \begin{columns}[T]
        \column{.4\textwidth}
            \begin{exampleblock}{\textattachfile{inheritance02.cpp}{Employee}}
                \vspace{-.2cm}
                \fontsize{7}{8} \selectfont
                \lstinputlisting[language=C++,linerange={3-15},numbers=left,firstnumber=3]{inheritance02.cpp}
                \vspace{-.2cm}
            \end{exampleblock}
        \column{.6\textwidth}
            \begin{exampleblock}{\textattachfile{inheritance02.cpp}{Programmer}}
                \vspace{-.2cm}
                \fontsize{7}{8} \selectfont
                \lstinputlisting[language=C++,linerange={17-34},numbers=right,firstnumber=17]{inheritance02.cpp}
                \vspace{-.2cm}
            \end{exampleblock}
    \end{columns}
\end{frame}

\begin{frame}[fragile]
    \begin{exampleblock}{\textattachfile{inheritance02.cpp}{main()}}
        \footnotesize
        \lstinputlisting[language=C++,linerange={36-40},numbers=left,firstnumber=36]{inheritance02.cpp}
    \end{exampleblock}
    \begin{block}{Kimenet}
        \footnotesize
        \vspace{-.3cm}
        \begin{verbatim}
300000
750000
        \end{verbatim}
        \vspace{-.6cm}
    \end{block}
\end{frame}

\begin{frame}[fragile]
    \begin{exampleblock}{\textattachfile{inheritance02.cpp}{main()}}
        \footnotesize
        \lstinputlisting[language=C++,linerange={42-48},numbers=left,firstnumber=42]{inheritance02.cpp}
    \end{exampleblock}
    \begin{block}{Kimenet}
        \footnotesize
        \vspace{-.3cm}
        \begin{verbatim}
300000
300000
        \end{verbatim}
        \vspace{-.6cm}
    \end{block}
\end{frame}

\begin{frame}
    Probléma:
    \begin{itemize}
        \item Az ős privát tagjainak elérése körülményes a leszármazottból.
    \end{itemize}
    Megoldás:
    \begin{itemize}
        \item A védett (\texttt{protected}) tagok elérhetők a saját osztályukban és minden leszármazottban.
        \item Származtatásnál is használható ez és a \texttt{private} kulcsszó, bár szinte mindig \texttt{public}-ot használunk, azaz nem korlátozzuk tovább az öröklött láthatósági kategóriákat.
    \end{itemize}
\end{frame}

\begin{frame}
    \begin{center}
        \includegraphics[scale=0.75]{inheritance03.eps}
      \end{center}
\end{frame}

\begin{frame}
    Probléma:
    \begin{itemize}
        \footnotesize
        \item Az ősre utaló típusú, de a leszármazott egyik példányát címző mutatóval csak az ős függvényét tudjuk hívni.
    \end{itemize}
    Megoldás:
    \begin{itemize}
        \footnotesize
        \item Amikor egy objektum tagfüggvényét hívjuk, mindig eldönthető, hogy az ős vagy a leszármazott felüldefiniált függvényéről van-e szó, és fordítási időben meghatározható annak címe (\emph{early/static bindig, korai/statikus kötés}). Pl. \texttt{Programmer p(300000, 2.5); p.getSalary();}
        \item Viszont egy objektum elérhető az ős típust címző mutatóval is (biztonságos, mert a leszármazott mindennel rendelkezik, amivel az őse). A korai kötés miatt az ősben definiált függvényt tudjuk csak elérni.
        \item A \emph{késői/dinamikus kötés} (late/dynamic binding) hatására a fordító \emph{virtuális metódustáblát} (VMT, vtable) hoz létre, melyben a leszármazottak felüldefiniált függvényeinek címei is benne vannak, melynek segítségével az aktuális objektum típusának megfelelő változat is hívható, típuskényszerítés nélkül. Pl.: \texttt{Employee* pe = new Programmer(); pe->getSalary();}
        \item C++-ban késői kötést \emph{virtuális tagfüggvényekkel} (\texttt{virtual}) hozhatunk létre.
    \end{itemize}
\end{frame}

\begin{frame}
    \begin{columns}[T]
        \column{.45\textwidth}
            \begin{exampleblock}{\textattachfile{inheritance03.cpp}{Employee}}
                \vspace{-.2cm}
                \fontsize{7}{8} \selectfont
                \lstinputlisting[language=C++,linerange={3-16},numbers=left,firstnumber=3]{inheritance03.cpp}
                \vspace{-.2cm}
            \end{exampleblock}
        \column{.55\textwidth}
            \begin{exampleblock}{\textattachfile{inheritance03.cpp}{Programmer}}
                \vspace{-.2cm}
                \fontsize{7}{8} \selectfont
                \lstinputlisting[language=C++,linerange={18-33},numbers=right,firstnumber=17]{inheritance03.cpp}
                \vspace{-.2cm}
            \end{exampleblock}
    \end{columns}
\end{frame}

\begin{frame}[fragile]
    \begin{exampleblock}{\textattachfile{inheritance03.cpp}{main()}}
        \footnotesize
        \lstinputlisting[language=C++,linerange={35-35},numbers=left,firstnumber=35]{inheritance03.cpp}
        \lstinputlisting[language=C++,linerange={41-46},numbers=left,firstnumber=41]{inheritance03.cpp}
    \end{exampleblock}
    \begin{block}{Kimenet}
        \footnotesize
        \vspace{-.3cm}
        \begin{verbatim}
300000
750000
        \end{verbatim}
        \vspace{-.6cm}
    \end{block}
\end{frame}

\subsection{Absztrakt osztályok}

\begin{frame}
    Induljunk ki \textattachfile{Rectangle12.cpp}{Rectangle12}-ből, majd 
    \begin{itemize}
        \item a kerület/terület lekérdezést (minden síkidomnál hasonlóan elvégezhető tevékenység) válasszuk külön a számítások elvégzésétől,
        \item a méretek legyenek módosíthatóak, és
        \item készítsünk háromszög (\texttt{Triangle}) osztályt is!
    \end{itemize}
\end{frame}    
     
\begin{frame}
    \begin{columns}[T]
        \column{.5\textwidth}
            \begin{exampleblock}{\textattachfile{Rectangle14.h}{Rectangle14.h}}
                \scriptsize
                \lstinputlisting[language=C++,linerange={4-10},numbers=left,firstnumber=4]{Rectangle14.h}
            \end{exampleblock}
        \column{.5\textwidth}
            \begin{exampleblock}{\textattachfile{Triangle14.h}{Triangle14.h}}
                \scriptsize
                \lstinputlisting[language=C++,linerange={6-11},numbers=right,firstnumber=6]{Triangle14.h}
            \end{exampleblock}
    \end{columns}
\end{frame}

\begin{frame}
    \begin{columns}[T]
        \column{.5\textwidth}
            \begin{exampleblock}{\textattachfile{Rectangle14.h}{Rectangle14.h}}
                \scriptsize
                \lstinputlisting[language=C++,linerange={11-22},numbers=left,firstnumber=11]{Rectangle14.h}
            \end{exampleblock}
        \column{.5\textwidth}
            \begin{exampleblock}{\textattachfile{Triangle14.h}{Triangle14.h}}
                \scriptsize
                \lstinputlisting[language=C++,linerange={12-23},numbers=right,firstnumber=12]{Triangle14.h}
            \end{exampleblock}
    \end{columns}
\end{frame}

\begin{frame}
    \begin{columns}[T]
        \column{.5\textwidth}
            \begin{exampleblock}{\textattachfile{Rectangle14.h}{Rectangle14.h}}
                \scriptsize
                \lstinputlisting[language=C++,linerange={23-33},numbers=left,firstnumber=23]{Rectangle14.h}
            \end{exampleblock}
        \column{.5\textwidth}
            \begin{exampleblock}{\textattachfile{Triangle14.h}{Triangle14.h}}
                \scriptsize
                \lstinputlisting[language=C++,linerange={24-34},numbers=right,firstnumber=24]{Triangle14.h}
            \end{exampleblock}
    \end{columns}
\end{frame}

\begin{frame}
    \begin{columns}[T]
        \column{.5\textwidth}
            \begin{exampleblock}{\textattachfile{Rectangle14.cpp}{Rectangle14.cpp}}
                \fontsize{7}{8} \selectfont
                \vspace{-.2cm}
                \lstinputlisting[language=C++,linerange={3-6},numbers=left,firstnumber=3]{Rectangle14.cpp}
                \lstinputlisting[language=C++,linerange={13-19},numbers=left,firstnumber=13]{Rectangle14.cpp}
                \lstinputlisting[language=C++,linerange={29-31},numbers=left,firstnumber=29]{Rectangle14.cpp}
                \vspace{-.2cm}
            \end{exampleblock}
        \column{.5\textwidth}
        \begin{exampleblock}{\textattachfile{Triangle14.cpp}{Triangle14.cpp}}
            \fontsize{7}{8} \selectfont
            \vspace{-.2cm}
            \lstinputlisting[language=C++,linerange={3-6},numbers=right,firstnumber=3]{Triangle14.cpp}
            \lstinputlisting[language=C++,linerange={18-24},numbers=right,firstnumber=18]{Triangle14.cpp}
            \lstinputlisting[language=C++,linerange={34-38},numbers=right,firstnumber=34]{Triangle14.cpp}
            \vspace{-.2cm}
        \end{exampleblock}
    \end{columns}
\end{frame}

\begin{frame}[fragile]
    \begin{exampleblock}{\textattachfile{main14.cpp}{main14.cpp}}
        \vspace{-.2cm}
        \fontsize{7}{8} \selectfont
        \lstinputlisting[language=C++,linerange={2-15},numbers=left,firstnumber=2]{main14.cpp}
        \vspace{-.2cm}
    \end{exampleblock}
    \begin{block}{Kimenet}
        \vspace{-.4cm}
        \fontsize{7}{8} \selectfont
        \begin{verbatim}
Rectangle #1 Area: 2 Perimeter: 6
Rectangle #2 Area: 6 Perimeter: 10
Rectangle #3 Area: 12 Perimeter: 14
        \end{verbatim}
        \vspace{-.6cm}
    \end{block}
\end{frame}

\begin{frame}[fragile]
    \begin{exampleblock}{\textattachfile{main14.cpp}{main14.cpp}}
        \vspace{-.2cm}
        \fontsize{8}{9} \selectfont
        \lstinputlisting[language=C++,linerange={16-27},numbers=left,firstnumber=16]{main14.cpp}
        \vspace{-.2cm}
    \end{exampleblock}
    \begin{block}{Kimenet}
        \vspace{-.4cm}
        \fontsize{8}{9} \selectfont
        \begin{verbatim}
Triangle #1 Area: 6 Perimeter: 12
Triangle #2 Area: 30 Perimeter: 30
Triangle #3 Area: 84 Perimeter: 56
        \end{verbatim}
        \vspace{-.6cm}
    \end{block}
\end{frame}

\begin{frame}
    Probléma:
    \begin{itemize}
        \item Sok ismétlődő kódrészlet $\to$ szükség lenne egy általános síkidomra a származtatáshoz, de az milyen adatokkal írható le? Hogyan számolható ki pl. a kerülete?
    \end{itemize}
    Megoldás:
    \begin{itemize}
        \item \emph{Absztrakt} (abstract) osztály: csak öröklési céllal hozzák létre, nem példányosítható a nem implementált, ún. \emph{pure virtual} tagfüggvények (absztrakt metódus) miatt
        \item Ha egy absztrakt osztály kizárólag pure virtual tagfüggvényeket tartalmaz: \emph{interfész} (interface) $\to$ elvárt szolgáltatáscsomag előírására
    \end{itemize}
\end{frame}

\begin{frame}
    \begin{center}
        \includegraphics[scale=0.6]{Shape15.eps} \\
      \end{center}
\end{frame}

\begin{frame}
    \begin{exampleblock}{\textattachfile{Shape15.h}{Shape15.h}}
        \lstinputlisting[language=C++,linerange={4-13},numbers=left,firstnumber=4]{Shape15.h}
    \end{exampleblock}
\end{frame}

\begin{frame}
    \begin{exampleblock}{\textattachfile{Shape15.h}{Shape15.h}}
        \lstinputlisting[language=C++,linerange={15-24},numbers=left,firstnumber=15]{Shape15.h}
    \end{exampleblock}
\end{frame}

\begin{frame}
    \begin{exampleblock}{\textattachfile{Shape15.cpp}{Shape15.cpp}}
        \lstinputlisting[language=C++,linerange={1-9},numbers=left,firstnumber=1]{Shape15.cpp}
    \end{exampleblock}
\end{frame}

\begin{frame}
    \begin{exampleblock}{\textattachfile{Rectangle15.h}{Rectangle15.h}}
        \vspace{-.3cm}
        \lstinputlisting[language=C++,linerange={4-15},numbers=left,firstnumber=4]{Rectangle15.h}
        \vspace{-.3cm}
    \end{exampleblock}
\end{frame}

\begin{frame}
    \begin{exampleblock}{\textattachfile{Rectangle15.h}{Rectangle15.h}}
        \lstinputlisting[language=C++,linerange={17-22},numbers=left,firstnumber=17]{Rectangle15.h}
    \end{exampleblock}
\end{frame}

\begin{frame}
    \begin{columns}[T]
        \column{.5\textwidth}
            \begin{exampleblock}{\textattachfile{Rectangle15.cpp}{Rectangle15.cpp}}
                \fontsize{7}{8} \selectfont
                \lstinputlisting[language=C++,linerange={1-6},numbers=left,firstnumber=1]{Rectangle15.cpp}
                \lstinputlisting[language=C++,linerange={13-15},numbers=left,firstnumber=13]{Rectangle15.cpp}
            \end{exampleblock}
        \column{.5\textwidth}
            \begin{exampleblock}{\textattachfile{Triangle15.cpp}{Triangle15.cpp}}
                \fontsize{7}{8} \selectfont
                \lstinputlisting[language=C++,linerange={1-6},numbers=right,firstnumber=1]{Triangle15.cpp}
                \lstinputlisting[language=C++,linerange={18-23},numbers=right,firstnumber=18]{Triangle15.cpp}
            \end{exampleblock}
    \end{columns}
\end{frame}

\begin{frame}
    \begin{exampleblock}{\textattachfile{main15.cpp}{main15.cpp}}
        \footnotesize
        \lstinputlisting[language=C++,linerange={1-14},numbers=left,firstnumber=1]{main15.cpp}
    \end{exampleblock}
\end{frame}

\begin{frame}[fragile]
    \begin{exampleblock}{\textattachfile{main15.cpp}{main15.cpp}}
        \footnotesize
        \vspace{-.3cm}
        \lstinputlisting[language=C++,linerange={15-21},numbers=left,firstnumber=15]{main15.cpp}
        \vspace{-.3cm}
    \end{exampleblock}
    \begin{block}{Kimenet}
        \footnotesize
        \vspace{-.3cm}
        \begin{verbatim}
9Rectangle #1 Area: 2 Perimeter: 6
9Rectangle #2 Area: 6 Perimeter: 10
8Triangle #3 Area: 6 Perimeter: 12
8Triangle #4 Area: 30 Perimeter: 30
        \end{verbatim}
        \vspace{-.6cm}
    \end{block}
\end{frame}

\begin{frame}[fragile]
    \begin{exampleblock}{\textattachfile{main15.cpp}{main15.cpp}}
        \footnotesize
        \vspace{-.3cm}
        \lstinputlisting[language=C++,linerange={22-30},numbers=left,firstnumber=22]{main15.cpp}
        \vspace{-.3cm}
    \end{exampleblock}
    \begin{block}{Kimenet}
        \footnotesize
        \vspace{-.3cm}
        \begin{verbatim}
Shape is abstract.
Rectangle is NOT abstract.
        \end{verbatim}
        \vspace{-.6cm}
    \end{block}
\end{frame}