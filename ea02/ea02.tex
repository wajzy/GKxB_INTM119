\documentclass[usenames,dvipsnames,aspectratio=169]{beamer}
\usepackage{../common/cpp}

\title[OO Programozás - C++]{OO Programozás}
\subtitle{Konstansok}

\begin{document}

%1
\begin{frame}[plain]
  \titlepage
  \logoalul
\end{frame}

\section{Konstansok}

\begin{frame}
    A \texttt{const} típusmódosító változókkal használva
    \begin{itemize}
        \item \emph{Konstans változó} (\kiemel{paradoxon!} $\to$ \emph{elnevezett konstans})
        \item Memóriában helyezik el, 
        \item értéke megjelenik a nyomkövető programokban (debugger),
        \item csak olvasható (fordító biztosítja),
        \item van típusa (vö. \texttt{\#define}).
        \item Láthatóságuk az őket definiáló állományra terjed csak ki.
    \end{itemize}
    \vfill
    Tömbök definiálása, méret megadása
    \begin{itemize}
        \item konstans kifejezéssel,
        \item konstans kifejezéssel inicializált elnevezett konstanssal,
        \item vagy tetszőleges kifejezéssel C99 / C++14-től
    \end{itemize}
\end{frame}

\begin{frame}
    \begin{exampleblock}{\textattachfile{constMain.cpp}{constMain.cpp}}
        \lstinputlisting[language=C++,linerange={1-11},numbers=left,firstnumber=1]{constMain.cpp}
    \end{exampleblock}
\end{frame}

\begin{frame}
    \begin{exampleblock}{\textattachfile{constMain.cpp}{constMain.cpp}}
        \small
        \lstinputlisting[language=C++,linerange={12-24},numbers=left,firstnumber=12]{constMain.cpp}
    \end{exampleblock}
\end{frame}

\begin{frame}
    \begin{exampleblock}{\textattachfile{constHeader.h}{constHeader.h}}
        \lstinputlisting[language=C++,linerange={1-3},numbers=left,firstnumber=1]{constHeader.h}
    \end{exampleblock}
    \begin{exampleblock}{\textattachfile{constSource.cpp}{constSource.cpp}}
        \lstinputlisting[language=C++,linerange={1-1},numbers=left,firstnumber=1]{constSource.cpp}
    \end{exampleblock}
\end{frame}

\end{document}