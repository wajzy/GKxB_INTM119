\documentclass[usenames,dvipsnames,aspectratio=169]{beamer}
\usepackage{../common/cpp}

\title[OO Programozás - C++]{OO Programozás}
\subtitle{Operátor felültöltés, másolás és átalakítás}

\begin{document}

%1
\begin{frame}[plain]
  \titlepage
  \logoalul
\end{frame}

\section{Operátorok felültöltése}

\subsection{Műveletek közönséges tagfüggvényekkel}

\begin{frame}
    Jelenlegi legjobb tudásunk szerint készítsünk osztályt egy \hiv{\href{https://hu.wikipedia.org/wiki/Komplex_sz\%C3\%A1mok}{komplex szám}} tárolására és műveletek elvégzésére, azaz:
    \begin{itemize}
        \item tároljuk a szám valós és képzetes részét, 
        \item készítsünk konstruktorokat,
        \item getter/setter tagfüggvényeket,
        \item és olyan függvényeket, melyekkel komplex számok összeadhatók és szorozhatók!
    \end{itemize}
\end{frame}

\begin{frame}
    \begin{exampleblock}{\textattachfile{Complex1.h}{Complex1.h}}
        \scriptsize
        \lstinputlisting[language=C++,linerange={6-21},numbers=left,firstnumber=6]{Complex1.h}
    \end{exampleblock}
\end{frame}

\begin{frame}
    \begin{exampleblock}{\textattachfile{Complex1.h}{Complex1.h}}
        \scriptsize
        \lstinputlisting[language=C++,linerange={22-37},numbers=left,firstnumber=22]{Complex1.h}
    \end{exampleblock}
\end{frame}

\begin{frame}
    \begin{exampleblock}{\textattachfile{main1.cpp}{main1.cpp}}
        \lstinputlisting[language=C++,linerange={1-10},numbers=left,firstnumber=1]{main1.cpp}
    \end{exampleblock}
\end{frame}

\subsection{Felültöltött operátorok tagfüggvényekkel}

\begin{frame}
    Operátor felültöltés
    \begin{itemize}
        \item A C++ megengedi, hogy az operátorok jelentését kiterjesszük a saját típusainkra (osztályainkra)
        \item Pl. ha értelmezhető az összeadás két \texttt{int} vagy \texttt{float} között, akkor két \texttt{Complex} objektum miért ne lehetne összeadható?
        \item Az operátorok működését (praktikusan nyilvános tag)függvények adják meg $\to$ \texttt{operatorX}, ahol \texttt{X} pl. \texttt{+}, \texttt{*}.
    \end{itemize}
\end{frame}

\begin{frame}
    \begin{exampleblock}{\textattachfile{Complex2.h}{Complex2.h}}
        \scriptsize
        \lstinputlisting[language=C++,linerange={31-37},numbers=left,firstnumber=31]{Complex2.h}
    \end{exampleblock}
    \begin{itemize}
        \small
        \item $+$ művelet bal operandusa: aktuális objektum, a jobb oldalit paraméterként kapja.
        \item Utóbbi felesleges (tagonkénti) másolásának elkerülésére referenciát használunk.
        \item Új, ideiglenes (eredmény) objektum jön létre e kettő alapján, ezt adja vissza.
    \end{itemize}
\end{frame}

\begin{frame}
    \begin{exampleblock}{\textattachfile{main2.cpp}{main2.cpp}}
        \footnotesize
        \lstinputlisting[language=C++,linerange={4-16},numbers=left,firstnumber=4]{main2.cpp}
    \end{exampleblock}
\end{frame}

\begin{frame}
    \begin{description}[mm]
        \item[Probléma:] \hfill \\ Összeadásnál a jobb oldali operandusnak \texttt{Complex}-nek kell lennie.
        \item[Megoldás:] \hfill \\ További felültöltött operátor függvények hozzáadása, pl. \texttt{double}-t hozzáadhatunk a valós részhez.
    \end{description}
    \begin{exampleblock}{\textattachfile{Complex3.h}{Complex3.h}}
        \lstinputlisting[language=C++,linerange={35-37},numbers=left,firstnumber=32]{Complex3.h}
    \end{exampleblock}
\end{frame}

\begin{frame}
    \begin{exampleblock}{\textattachfile{main3.cpp}{main3.cpp}}
        \small
        \lstinputlisting[language=C++,linerange={4-14},numbers=left,firstnumber=4]{main3.cpp}
    \end{exampleblock}
\end{frame}

\subsection{Felültöltött operátorok barát függvényekkel}

\begin{frame}
    \begin{itemize}
        \item Ez sem segít, ha a \emph{bal} operandus \texttt{double} típusú $\to$ \emph{nem tag}, két paraméteres \texttt{operator} függvény.
        \item Nem éri el a privát/védett tagokat $\to$ barát (\texttt{friend}) függvény: mindenhez hozzáfér az osztályon belül.
        \item Tagobjektum is lehet az osztályunk barátja, pl.: \texttt{friend class FriendOfComplex;}
    \end{itemize}
\end{frame}

\begin{frame}
    \begin{exampleblock}{\textattachfile{Complex4.h}{Complex4.h}}
        \scriptsize
        \lstinputlisting[language=C++,linerange={6-8},numbers=left,firstnumber=6]{Complex4.h}
        \lstinputlisting[language=C++,linerange={28-36},numbers=left,firstnumber=36]{Complex4.h}
    \end{exampleblock}
\end{frame}

\begin{frame}
    \begin{exampleblock}{\textattachfile{Complex4.cpp}{Complex4.cpp}}
        \footnotesize
        \lstinputlisting[language=C++,linerange={3-10},numbers=left,firstnumber=3]{Complex4.cpp}
    \end{exampleblock}
\end{frame}

% main4.cpp
% minden forgatókönyv lefedve, de körülményesen
% mi az az ostream?

\end{document}